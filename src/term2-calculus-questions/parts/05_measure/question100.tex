\Subsection{Билет 100: Лемма про дизъюнктное объединение
множеств. Кольцо и полукольцо. Теорема о свойствах элементов полукольца.}

\begin{lemma}\thmslashn
	
	$\bigcup\limits_{k=1}^{n} A_k = \bigsqcup\limits_{k=1}^{n} A_k \setminus (\bigcup\limits_{j=1}^{k-1} A_j)$
	
	
	$\bigcup\limits_{k=1}^{\infty} A_k = \bigsqcup\limits_{k=1}^{\infty} A_k \setminus (\bigcup\limits_{j=1}^{k-1} A_j)$
\end{lemma}

\begin{proof}\thmslashn
	
	Рассмотрим $B_k := A_k \setminus  (\bigcup\limits_{j=1}^{k-1} A_j)$

	Заметим, что $B_k \subset A_k$, поэтому если $i < k$, то $B_k \cap A_i = \emptyset \implies B_k \cap B_i = \emptyset$
	
	$\implies B_k$ --дизъюнктны.
	
	А ещё из того, что $B_k \subset A_k$, следует
	
	$\bigcup\limits_{k=1}^{?} B_k  \subset \bigcup\limits_{k=1}^{?} A_k$

	где $?$ означает либо $n$, если хотим доказать для конечного, либо $\infty$, если хотим доказать для счетного.
	
	обратное включение:
	
	Возьмем $x \in \bigcup\limits_{k=1}^{?} A_k $. Надо доказать, что он лежит и в объединении $B$. Для этого рассмотрим такой самый первый номер $m$, что $x\in A_m$. Но тогда он не лежит в $A_1,...A_{m-1}$, но именно эти мн-ва мы исключаем в $B_m$. Поэтому, $x$ будет лежать в $B_m$

\end{proof}



\begin{definition}\thmslashn
	
	$\mathcal{R}$ -- кольцо, если $\forall A,B \in \mathcal{R}$
	
	$\implies A\cap B ,\; A\cup B,\; A\setminus B\;\in \mathcal{R}$
\end{definition}

\begin{remark}\thmslashn
	
	Любая алгебра является кольцом. Это видно из определений. У кольца оно более слабое.
	
\end{remark}

\begin{remark}\thmslashn
	
	Если в кольце есть $X$, то это алгебра. Действительно, тогда берём любое $B$ из алгебры и $A=X$, получаем, что $X/B$ лежит $\implies$ симметричность. Пустое тоже есть, т.к. симметрично $X$.

	
\end{remark}

Таким образом, алгебра от кольца отличается только наличием $X$.


\begin{definition}\thmslashn
	
	$\mathcal{P}$ -- полукольцо, если 
	
	\begin{enumerate}
		\item $\emptyset \in \mathcal{P}$
		
		\item $\forall A,B \in \mathcal{P} \implies A\cap B \in \mathcal{P}$
		
		\item $A, B \in \mathcal{P} \implies$ существует конечное число дизъюнктных множеств $C_1,...,C_n$ из $\mathcal{P}$, т.ч. $A\setminus B = \bigsqcup\limits_{k=1}^{n} C_k$.
	\end{enumerate}
	
	
\end{definition}

\begin{example}\thmslashn
	
	\begin{enumerate}
		\item 
		Возьмём прямую и полуинтервалы на ней
		$X = \R \;\; \mathcal{P} = \{[a,b) \;\;:\;\; a\le b,\;\;a,b \in \R\}$
	
		$\mathcal{P}$ -- полукольцо.

		Действительно, пересечение полуинтервалов - полуинтервал. А вот разность может дать два полуинтервала (если один вложен в другой). Но третье условие нам как раз такое и разрешает.

		\item 
			Аналогично, но точки - рациональные. Не знаю, зачем Храбров дал этот пример, у Ани его нет. Док-во что полукольцо 1 в 1.
		\item 
			$X = \R^2 \;\; \mathcal{P} = \{[a,b)\times [c,d) \;\;:\;\; a\le b,\;\; c\le d,\;\; a,b \in \R\}$
			
			Пересечение двух прямоугольников - прямоугольник. А вот с разностью не так очевидно, если один прямоугольник лежит внутри другого. Но в этом случае можно продлить сторны одного до пересечения с другим и на получившиеся прямоугольники и разбить.

	\end{enumerate}
\end{example}

\begin{theorem}\thmslashn
	
	\begin{enumerate}
		\item $P_1, ...,P_n,P \in \mathcal{P} \implies \exists Q_1,...,Q_m \in \mathcal{P}$, т.ч.
		
		$P\setminus \bigcup\limits_{k=1}^{n} P_k = \bigsqcup\limits_{k=1}^{m} Q_k$
		
		\item $P_1, P_2,...,P_n \in \mathcal{P} \implies \exists Q_{jk} \in \mathcal{P}$, т.ч.
		
		
		$\bigcup\limits_{k=1}^{n} P_k = \bigsqcup\limits_{j=1}^{n}\bigsqcup\limits_{k=1}^{m} Q_{jk}$, причем $Q_{jk} \subset P_j$
		
		\item аналогично для $n = +\infty$.
	\end{enumerate}
\end{theorem}

\begin{proof}\thmslashn
	
	\begin{enumerate}
		\item Индукция по $n$. База -- определение полукольца.
		
		Переход $n \to n+1$:
		
		
		$P\setminus\bigcup\limits_{k=1}^{n+1}P_k = (P \setminus \bigcup\limits_{k=1}^n P_k)\setminus P_{n+1} = (\bigsqcup\limits_{k=1}^m Q_k) \setminus P_{n+1} = \bigsqcup\limits_{k=1}^m Q_k \setminus P_{n+1} = \bigsqcup\limits_{k=1}^m \bigsqcup\limits_{j=1}^{m_k} Q_{kj}$

		Первый знак равно - св-ва объединения. Второй - применяем индукционное предположение (в скобках). Третий - скобки можно снять. Четвёртый - третье условие в определении полукольца.
		
		\item Применим лемму о дизъюнктных объединениях (см. начало билета) и уже доказанный первый пункт:
		$\bigcup\limits_{k=1}^n P_k = \bigsqcup\limits_{k=1}^n P_k \setminus(\bigcup\limits_{j=1}^{k-1}P_j) = \bigsqcup\limits_{k=1}^n \bigsqcup\limits_{j=1}^{m_k} Q_{kj}$
		
		
		Осталось понять, что $Q_{kj} \subset P_k$. Но это верно, так как $Q_{kj} \subset P_k \setminus(\bigcup\limits_{j=1}^{k-1}P_j)$ 

		\item Аналогично, но используем лемму о дизъюнктных объединениях в форме для бесконечного числа множеств.
	\end{enumerate}
\end{proof}


