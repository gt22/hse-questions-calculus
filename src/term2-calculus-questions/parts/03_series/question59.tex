\Subsection{Билет 59: ! Признак сравнения. Признак Вейерштрасса. Следствие. Примеры.}

\begin{theorem}[Признак сравнения] \thmslashn

$u_n, v_n : E \mapsto \mathbb{R}$. Если $\forall x \in E$ $|u_n(x)| \leqslant v_n(x)$ и $\sum\limits_{n = 1}^{\infty}{v_n(x)}$ - равномерно сходится. Тогда $\sum\limits_{n = 1}^{\infty}{u_n(x)}$ - равномерно сходится.

\begin{proof} \thmslashn

  Так как $\sum\limits_{n = 1}^{\infty}{v_n(x)}$ - равномерно сходится, можно применить признак Коши:
  \[\begin{aligned}
    \forall \eps > 0\, \exists N\in\mathbb{N} : \forall n \geqslant N\, \forall p \in \mathbb{N}\, \forall x \in E \text{ выполняется } \left|\sum\limits_{k = n}^{n + p} v_k(x)\right| < \eps
  \end{aligned}\]
  Из условия теоремы можно записать неравенство на частичные суммы.
  \[\begin{aligned}
    \eps > \left|\sum\limits_{k = n}^{n+p}{v_k(n)}\right| \geqslant \sum\limits_{k = n}^{n+p}{|u_k(x)|} \geqslant \left|\sum\limits_{k = n}^{n+p}{u_k(x)}\right|
  \end{aligned}\]
  Получили, что критерий Коши выполняется для ряда $\sum\limits_{n = 1}^{\infty}{u_n(x)}$, значит он сходится равномерно.

\end{proof}
\end{theorem}

\begin{theorem}[Признак Вейерштрасса] \thmslashn

  $u_n : E \mapsto \mathbb{R}$. Если $\exists \{a_n\} :  |u_n(x)| \leqslant a_n \, \forall x \in E \text{ и } \sum\limits_{n = 1}^{\infty}{a_n}$ - сходится. Тогда $\sum\limits_{n=1}^{\infty}{u_n(x)}$ - равномерно сходится.
  \begin{proof} \thmslashn
    
    $v_n(x):=a_n \Rightarrow \sum\limits_{n=1}^{\infty}v_n$ -  равномерно сходится $\Rightarrow \sum\limits_{n=1}^{\infty}{u_n}$ -  равномерно сходится по признаку сравнения.
  \end{proof}
\end{theorem}

\begin{consequence} \thmslashn

  Если ряд $\sum\limits_{n = 1}^{\infty}{|u_n(x)|}$ - равномерно сходится, тогда $\sum\limits_{n = 1}^{\infty}{u_n(x)}$ - равномерно сходится.
  \begin{proof} \thmslashn

    Воспользуемся признаком сравнения для рядов $\sum\limits_{n = 1}^{\infty}{u_n(x)} \text{ и } \sum\limits_{n = 1}^{\infty}{|u_n(x)|}$
  \end{proof}
\end{consequence}

\begin{example} \thmslashn

  $\sum\limits_{n = 1}^{\infty}{\frac{\sin{x}}{n^2}}$ - равномерно сходится на $\mathbb{R}$
  \begin{proof} \thmslashn

    $\{a_n\} := \frac{1}{n^2}$. Воспользуемся признаком Вейерштрасса для $\sum\limits_{n = 1}^{\infty}{\frac{\sin{x}}{n^2}} \text{ и } a_n$
  \end{proof}
\end{example}

\begin{remark} \thmslashn

  Абсолютная и равномерная сходимости - разные вещи.

  \begin{enumerate}
  \item
    Ряд сходится абсолютно, но не сходится равномерно.
    \begin{example} \thmslashn

      $\sum\limits_{n = 1}^{\infty}{x^n}$ на $(-1; 1)$
      \begin{proof} \thmslashn

        $\sum\limits_{n = 1}^{\infty}{|x^{n}|} = \frac{1}{1 - |x|}$  - геометрическая прогрессия. Действительно, такой ряд сходится абсолютно. По критерию Коши докажем, что равномерной сходимости нет.
        \[\begin{aligned}
          \exists \eps > 0 : \exists N \in \mathbb{N}\, \forall n \geqslant N \,\exists p \in \mathbb{N} \,\exists \overline{x} \in E: \,\left|\sum\limits_{k = n}^{n+p}{v_n(\overline{x})}\right| \geqslant \eps
        \end{aligned}\]
        При выполнении такого условия равномерной сходимости не будет. Возьмем $\eps = \frac{1}{2}$, $p = 1$, $\overline{x} = \sqrt[n]{\frac{1}{2}}$.
      \end{proof}
    \end{example}
  \item
    Ряд сходится равномерно, но не сходится абсолютно.
    \begin{example} \thmslashn

      $\sum\limits_{n = 1}^{\infty}{\frac{(-1)^n}{n}}$ - сходится равномерно, но нет абсолютной сходимости.
    \end{example}
  \item
    Также бывает, что ряд сходится абсолютно, равномерно, но ряд из модулей не сходится равномерно.
    Пример в следующем вопросе. $\sum\limits_{n = 1}^{\infty}{(-1)^n\frac{x^n}{n}}$
  \end{enumerate}
\end{remark}


