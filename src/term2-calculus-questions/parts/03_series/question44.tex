\Subsection{Билет 44: Признак Даламбера. Примеры. Связь между признаками Коши и Даламбера.}

\begin{theorem}[признак Даламбера]\slashns
	
	$a_n > 0$
	
	\begin{enumerate}
		\item Если $\frac{a_{n+1}}{a_n} \le d < 1$, то ряд сходится.
		\item Если $\frac{a_{n+1}}{a_n} \ge 1$, то ряд расходится.
		\item $d^* = \lim\limits_{n \to \infty} \frac{a_{n+1}}{a_n}$
		
		Если $d^* < 1$, то ряд сходится.
		
		Если $d^* > 1$, то ряд расходится.
		
	\end{enumerate}
\end{theorem}

\begin{proof}\slashns
	
	\begin{enumerate}
		\item $a_n = \frac{a_n}{a_{n-1}} \cdot \frac{a_{n-1}}{a_{n-2}}\cdot ... \cdot \frac{a_2}{a_1} \cdot a_1$
		
		$\implies a_n \le d^{n-1}a_1$
		
		Ряд, мажорирующий геометрической прогрессией.
		
		\item $\frac{a_{n+1}}{a_n} \ge 1 \implies a_n \le a_{n+1} \implies$ 
		
		$0 < a_1 \le a_2 \le ...$
		
		$\implies a_n \not\to 0 \implies$ расходится
		
		\item $d^* < 1$
		
		$d:= \frac{d^*+1}{2} < 1$
		
		$\implies$ начиная с некоторого номера $\frac{a_{n+1}}{a_n} < d < 1 \implies$ попали в первый пункт, сходится
		
		$d^* > 1$
		
		$\implies$ с некоторого номера $\frac{a_{n+1}}{a_n} \ge 1$
		
		$\implies$ ряд расходится. 
	\end{enumerate}
\end{proof}

\begin{example}\slashns
	
	$\sum\limits_{n = 0}^{\infty} \frac{x^n}{n!}$, при $x > 0$
	
	По Даламберу $\frac{a_{n+1}}{a_n} = \frac{x^{n+1}}{(n+1)!} \cdot \frac{n!}{x^n} = \frac{x}{n+1} \to 0$
	
	$\implies $ ряд сходится
	
	По Коши.
	
	$\sqrt[n]{\frac{x^n}{n!}} = \frac{x}{\sqrt[n]{n!}} \sim \frac{xe}{n} \to 0$
\end{example}

\begin{theorem}\slashns
	
	$a_n > 0$
	
	Если существует $\lim\limits_{n \to \infty} \frac{a_{n+1}}{a_n} =: d^*$, то существует и $\lim\limits_{n \to \infty} \sqrt[n]{a_n}$ и он равен $d^*$
\end{theorem}

\begin{proof}\slashns
	
	Применяем Штольца!
	
	$\lim\limits_{n \to \infty} \ln \sqrt[n]{a_n} = \lim\limits_{n \to \infty} \frac{\ln a_n}{n} = \lim\limits_{n \to \infty} \frac{\ln a_{n+1} - \ln a_n}{(n+1) - n} = \lim\limits_{n \to \infty} \ln (\frac{a_{n+1}}{a_n}) = \ln (\lim\limits_{n \to \infty} \frac{a_{n+1}}{a_n}) = \ln d^*$
	
\end{proof}

