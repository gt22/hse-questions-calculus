\Subsection{Билет 71: ! Определение $e^z$, $\sin{z}$ и $\cos{z}$. Свойства. Ряд Тейлора для $\log(1 + x)$.}

\begin{theorem} [Определение и свойства $e^z$] \thmslashn
    
    Мы уже знаем разложение в ряд Тэйлора:
    
    \[e^{x}=\sum_{n=0}^{\infty} \frac{x^{n}}{n!}\]
    
	Показательную функцию комплексного переменного 
	
	\[w = e^{z}\]
	
    определим равенством
    
    \[e^{z}=e^{(x+iy)}=e^{x}(\cos{y} + i\sin{y}).\]
    
    Справедлива формула Эйлера:
    
    \[e^{iy}=\cos{y} + i\sin{x}\]
    
    Функция $w = e^{z}$ определена на всей комплексной плоскости и на действительной оси совпадает с соответствующей функцией действительного переменного. Сходится во всей плоскости.
    
    Свойства:
    
    \begin{itemize}
        \item
        
            $e^{z_{1}}e^{z_{2}}=e^{z_{1}+z_{2}}$;
            
        \item
        
            $e^{z}\neq0, \text{ так как } |e^{z}| =  e^{x} > 0$;
            
        \item
            
            $e^{z} \text{периодическая с периодом } T = 2\pi i \text{, т.е. } e^{z+2\pi i} = e^{z}$
            
    \end{itemize}
    
\end{theorem}

\begin{theorem} [Определение и свойства $\sin{z}$ и $\cos{z}$] \thmslashn

    Мы уже знаем разложение в ряд Тэйлора:
    
    \[\cos{x}=\sum_{n=0}^{\infty} \frac{(-1)^{n}x^{2n}}{2n!} \text{,  } \sin{x}=\sum_{n=0}^{\infty} \frac{(-1)^{n}x^{2n+1}}{(2n+1)!}\]
    
	Тригонометрические функции $\sin{z}$ и $\cos{z}$ определим через показательную функцию по формулам Эйлера
	
	\[\sin{z}=\frac{e^{iz} - e^{-iz}}{2i} \text{,   } \cos{z}=\frac{e^{iz} + e^{-iz}}{i}\]
    
    Сходятся во всей плоскости.
    
    Свойства:
    
    \begin{itemize}
        \item
        
            При $z = x, \sin{z} \text{ и } \cos{z}$ совпадают с тригонометрическими функциями $\sin{x} \text{ и } \cos{x}$ действительной переменной $x$.
            
        \item
        
            Выполняются основные тригонометрические соотношения.
            
        \item
            
            $\sin{z} \text{ и } \cos{z}$ периодические функции с основным периодом $2\pi$.
            
        \item
            
            $\sin{z}$ - нечетная функция, $\cos{z}$ - четная функция.
                
        \item
                
            Могут принимать любые значения, а не только ограниченные по модулю единицей.
                
        \item
        
            $\cos^{2}z + \sin^{2}z = 1$.
            
    \end{itemize}
    
\end{theorem}

\begin{theorem} [Ряд Тейлора для $\log(1 + x)$] \thmslashn

    \[\log(1+x) = \sum_{n=1}^{\infty} \frac{(-1)^{n-1}x^{n}}{n} \text{, при } x \in (-1, 1)\]
    
    \begin{proof} \thmslashn
        \[\frac{1}{1+x} =  \sum_{n=0}^{\infty} (-1)^{n}x^{n}\text{, при } x \in (-1, 1)\]
        
        
        \[\int_{0}^{x} \frac{dt}{1 + t} = \int_{0}^{x} \sum_{n=0}^{\infty}(-1)^{n}t^{n} dt = \sum_{n=0}^{\infty}(-1)^{n} \int_{0}^{x} t^{n}dt =  \sum_{n=0}^{\infty}(-1)^{n}\frac{x^{n+1}}{n+1} =  \sum_{k=1}^{\infty}(-1)^{k} \frac{x^{k}}{k} \text {, } k=n+1\]
    \end{proof}
    
\end{theorem}
