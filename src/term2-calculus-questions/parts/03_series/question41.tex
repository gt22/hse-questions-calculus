\Subsection{Билет 41: Критерий Коши. Абсолютная сходимость. Группировка членов ряда. Свойства}

\begin{theorem}[Критерий Коши] \thmslashn 
	
	$X$ -- полное нормированное пространство.
	
	$\sum a_n$ сходится $\iff $ $\forall \varepsilon > 0\;\; \exists N \;\; \forall m>n>N \;\; \|\sum\limits_{k=n}^{m}a_k\| < \varepsilon$

	\begin{proof} \thmslashn
		
		$S_n:= \sum\limits_{k=1}^{n} a_k$
		
		$\sum a_n$ -- сходится $\iff \exists $ конечный $\lim\limits_{n \to \infty} S_n$
		
		$\iff$ (полнота $X$) $S_n$ -- фундаментальная последовательность
		
		$\iff \; \forall \varepsilon > 0 \;\; \exists N \;\; \forall m,n > N\; \|S_m - S_n\| < \varepsilon$
	
	$\|S_m - S_{n - 1}\| = \|\sum\limits_{k =n}^{m}a_k\|$
	\end{proof}

\end{theorem}

\begin{definition}[Абсолютная сходимость] \thmslashn 
	
	$x_n \in X$ -- нормированное пространство

	$\sum\limits_{n = 1}^{\infty}x_n$ -- абсолютно сходится, если $\sum\limits_{n = 1}^{\infty}\|x_n\|$ -- сходится

\end{definition}

\begin{theorem} \thmslashn 

	$X$ -- полное нормированное пространство

	Если $\sum\limits_{n=1}^{\infty}x_n$ -- абсолютно сходится, то
	\begin{enumerate}
		\item $\sum\limits_{n=1}^{\infty}x_n$ -- сходится 
		
		\begin{proof} \thmslashn 

			$\sum\limits_{n=1}^{\infty}\|x_n\|$ -- сходится $\implies$ (Критерий Коши для $\|x_n\|$) $\;\forall \varepsilon\; \exists N \;\; m,n\ge N \; \sum\limits_{k=n}^{m}\|x_k\|< \varepsilon$

			$\varepsilon > \sum\limits_{k=n}^{m}\|x_k\| \ge \|\sum\limits_{k=n}^{m}x_k\|$
	
			$\implies\; \forall \varepsilon \; \exists N \; \forall n,m \ge N\; \|\sum\limits_{k=n}^{m}x_k\| < \varepsilon$
			
			$\implies$ (Критерий Коши для $x_n$) $\sum\limits_{n=1}^{\infty}x_n$ -- сходится
		\end{proof}

		\item $\|\sum\limits_{n=1}^{\infty}x_n\| \le \sum\limits_{n=1}^{\infty}\|x_n\|$
		
		\begin{proof} \thmslashn 
			\[
				\|\sum\limits_{k = 1}^{n}x_k\| \le \sum\limits_{k = 1}^{n}\|x_k\|
			\]
			\[
				\|\sum\limits_{k = 1}^{n}x_k\| \to \|\sum\limits_{k = 1}^{\infty}x_k\|\;\text{и}\;\sum\limits_{k = 1}^{n}\|x_k\|\to\sum\limits_{k = 1}^{\infty}\|x_k\|
			\]
			\[
				\implies \|\sum\limits_{k = 1}^{\infty}x_k\| \le \sum\limits_{k = 1}^{\infty}\|x_k\|
			\]
		\end{proof}
	\end{enumerate} 
\end{theorem}

\begin{definition}[Группировка членов ряда] \thmslashn 

	$(x_1 + x_2) + (x_3 + x_4 + x_5) + x_6 + (x_7 + x_8)+\ldots$
\end{definition}

\begin{remark}\thmslashn

	\begin{enumerate}
		\item Если исходный ряд сходился, то ряд получившийся после группировки сходится к той же сумме.
		\item В обратную сторону верно не всегда
		\begin{example}
			$(1 - 1)+(1- 1) + (1-1)+\ldots$
		\end{example}
	\end{enumerate}
\end{remark}

\begin{theorem}[Когда верно в обратную сторону] \thmslashn

	$\sum\limits_{n=1}^{\infty}x_n$

	$S_n = \sum\limits_{k=1}^{n}x_k$

	\begin{enumerate}

		\item Если $\lim x_n = 0$ и количество слагаемых в каждой группе $\le M$
		
			\begin{proof} \thmslashn

				$S_{n_k}$ -- подпоследовательность чатичных сумм $\lim\limits_{k\to \infty} S_{n_k} = S$

				(группировка – всего лишь выбор подпоследовательности частичных сумм)

				\[
					x_1 + x_2 + \ldots + x_{n_k}) + (x_{n_k + 1} + \ldots + x_{n_k + r} + \ldots	
				\]

				\[
					\|S_{n_k + r} - S\| = \|S_{n_k} - S + x_{n_k + 1} + \ldots + x_{n_k + r}\| \le	
				\]

				\[
					\le \|S_{n_k} - S\| + \|x_{n_k + 1}\| + \ldots + \|x_{n_k + r}\|	
				\]

				Выберем $K$, т.ч. если $k \ge K$, то $\|S_{n_k} - S\| < \varepsilon$

				Выберем $N$, т.ч. если $n \ge N$, то $\|x_n\| <  \varepsilon$

				Если выполняется и то, и то, тогда:
				
				$\|S_{n_k} - S\| + \|x_{n_k + 1}\| + \ldots + \|x_{n_k + r}\| < \varepsilon(M + 1)$

				Значит $\forall \varepsilon > 0$ мы можем выбрать $N_1$, т.ч. $\forall n \ge N_1 \; \|S_n - S\| < \varepsilon$
			\end{proof}
		\item Для числовых рядов. Если все члены ряда в группе одного знака.

			\begin{proof} \thmslashn
				
				$S_{n_k} \to S$

				$\forall \varepsilon > 0\; \exists K \; \forall k \ge K\;\; |S_{n_k} - S| < \varepsilon$
		
				$N:= n_K$
		
				если $n \ge N$:
				
				для некоторого $k$: \;$n_k \le n < n_{k+1}$
		
				$S_n = S_{n_k} + x_{n_k+1} + x_{n_k+2} + ...+x_n$
		
				если в группе все члены $\ge 0$, то $S_n \ge S_{n_k}$
		
				$S_n = S_{n_{k+1}} - x_{n_{k+1}} - x_{n_{k+1} - 1} - ... - x_{n+1}$
		
				$S_{n_{k+1}} \ge S_n$
		
				Тогда $|S_n - S| < \varepsilon$
		
				Если в группе отрицательные члены
		
				$S_{n_k} \ge S_n \ge S_{n_{k+1}}$
		
				Тогда в этом случае тот же вывод

				$|S_n - S| < \varepsilon$

			\end{proof}
	\end{enumerate}

\end{theorem}