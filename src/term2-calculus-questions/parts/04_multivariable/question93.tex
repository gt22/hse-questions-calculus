\Subsection{Билет 93: ! Локальные экстремумы. Определение и необходимое условие экстремума. Стационарные точки.}
\begin{definition} \thmslashn

	$E \subset \mathbb{R}^n,\ f : E \mapsto \mathbb{R}, a \in E$
	
	$a$ -- точка локального минимума, если $\exists U$ -- окрестность точки $a$, такая что $\forall x \in U \cap E\ f(x) \ge f(a)$.
	
	$a$ -- точка строгого локального минимума, если $\exists U$ -- окрестность точки $a$, такая что $\forall x \in U \cap E,\\x \neq a\ f(x) > f(a)$. 
	
	Аналогично определяются локальный максимум, строгий локальный максимум, точка экстремума и точка строгого экстремума.
\end{definition}

\begin{theorem}[Необходимое условие экстремума] \thmslashn

	$f : E \mapsto \mathbb{R}, a \in \Int E, a$ -- точка экстремума. Если существует $\frac{\partial f}{\partial x_k}(a)$, то $\frac{\partial f}{\partial x_k}(a) = 0$. В частности, если $f$ -- дифференцируема в точке $a$, то $\frac{\partial f}{\partial x_1}(a) = \dots = \frac{\partial f}{\partial x_n}(a) = 0$, т.е. $\nabla f(a) = 0$.
	\begin{proof} \thmslashn
		
		Пусть $a$ -- точка минимума для $f$. Заведем функцию $g(t) = f(a_1, \dots, a_{k - 1}, t, a_{k + 1}, \dots, a_n) \ge f(a_1, a_2, \dots, a_n) = g(a_k)$. Тогда $a_k$ -- точка минимума для $g$ и $g$ дифференцируема в точке $a_k$. Применяем одномерную теорему и получается, что $g'(a_k) = 0$, но $\frac{\partial f}{\partial x_k}(a) = g'(a_k) = 0$.
	\end{proof}
\end{theorem}

\begin{definition}[Стационарная точка] \thmslashn
	
	Если $f$ дифференцируема в точке $a$ и $\nabla f(a) = 0$, то $a$ -- стационарная точка.
\end{definition}

\begin{remark}[Формула Тейлора в стационарной точке] \thmslashn

	\[
	f(a + h) = f(a) + \frac{1}{2}\sum_{i, j}\frac{\partial^2f}{\partial x_i \partial x_j}(a)h_ih_j + o(\|h\|^2)
	\]
\end{remark}
