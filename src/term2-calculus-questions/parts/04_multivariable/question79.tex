\Subsection{Билет 79: Теорема Лагранжа для векторнозначных функций.}

\begin{theorem} \thmslashn

    $f : [a, b] \mapsto \mathbb{R}^m$ непрерывна и дифференцируема на $(a, b)$. Тогда $\exists c \in (a, b)$, такая что $\left\Vert f(b) - f(a) \right\Vert \leqslant \left\Vert f'(c)\right\Vert(b - a)$
    \begin{proof} \thmslashn

	$\varphi(x) := \langle f(x), f(b) - f(a)\rangle : [a, b] \mapsto \mathbb{R}$\newline
	$\varphi(x)$ удовлетворяет условию одномерной теоремы Лагранжа\newline
	$\exists c \in (a, b)$, т.ч. $\varphi(b) - \varphi(a) = \varphi(c')(b - a) = \langle f'(c), f(b) - f(a)\rangle(b - a)$\newline
	$\varphi'(x) = \langle f'(x), f(b) - f(a)\rangle + \langle f(x), (f(b) - f(a))'\rangle = \langle f'(x), f(b) - f(a)\rangle$\newline
        $\varphi(b) - \varphi(a) = \langle f(b), f(b) - f(a)\rangle - \langle f(a), f(b) - f(a)\rangle = \langle f(b) - f(a), f(b) - f(a)\rangle = \left\Vert f(b) - f(a) \right\Vert^2$\newline 
        $\left\Vert f(b) - f(a) \right \Vert^2 = \langle f'(c), f(b) - f(a)\rangle(b - a) \leqslant \left \Vert f'(c) \right\Vert \left \Vert f(b) - f(a) \right \Vert (b - a)$ (Коши-Буняковский)\newline
        $\left\Vert f(b) - f(a)\right\Vert \leqslant \left\Vert f'(c) \right\Vert(b - a)$
    \end{proof}
    \item Замечание. Равенство может никогда не достигаться \newline
	$f(x) = (\cos{x}, \sin{x}) : [0, 2\Pi] \mapsto \mathbb{R}^2$\newline
	$f(0) = (1, 0) = f(2\Pi)$\newline
	$f(2\Pi) - f(0) = (0, 0) \Rightarrow \left\Vert f(2\Pi) - f(0) \right\Vert = 0$\newline
	$f'(x) = ((\cos{x})', (\sin{x})') = (-\sin{x}, \cos{x})\newline
	\left\Vert f'(x) \right\Vert = 1 \Rightarrow \left\Vert f'(c) \right\Vert (2\Pi - 0) = 2\Pi > \left\Vert f(2\Pi) - f(0) \right\Vert = 0$

\end{theorem}

