\Subsection{Билет 90: Дифференцируемость обратного отображения. Образ области при невырожденном отображении}

Напоминание теоремы об обратной функции:

\begin{theorem} \thmslashn

    $f:D\rightarrow R^{n}, D \subset R^{n} \text{ открытое, } x_{0} \in D, f \text{ непрерывно дифференцируема в окрестности } (\cdot) x_{0} \text{ и } y_{0}=\text{ }=f(x_{0}), \text{ матрица } A:=f'(x_{0}) \text{ обратима. Тогда существуют окрестности } U \text{ точки } x_{0}, V \text{ окрестность } $ $(\cdot) y_{0} \text{, т.ч. } f:U\rightarrow V - \text{ обратима и } f^{-1}:V\rightarrow U - \text{ непрерывна.}$
\end{theorem}



\begin{theorem}[Теорема о дифференцируемости обратного отображения] 
    $f: D \to \R^n, D \subset \R^N$ -- открытое. $f$ непрерывна. $a \in D, f(a) = b$. $f$ дифференцируема в точке $a$, $U$ -- окрестность точки $a$, $V$ -- окрестность точки $b$, $A := f'(a)$ -- обратима, $f^{-1}: V \to U$ -- непрерывна. Тогда $g := f^{-1}$ -- дифференцируема в точке $b$.
\end{theorem}

\begin{proof} \thmslashn

    Определение дифференцируемости $f$ в $a$:

    $f(a+h) = f(a) + Ah + \alpha(h)\|h\|$, где $\alpha(h) \to 0$ при $h \to 0$

    Обозначим $K := f(a+h) - f(a) = Ah + \alpha(h)\|h\|$

    Хотим (Храбров хочет) доказать, что если $K \to 0$, то $h \to 0$. Зачем - станет понятно позже

    \[
        \|h\| = \|A^{-1} A h \| \le \|A^{-1}\| \| Ah \| \implies \|Ah\| \ge \frac{\|h\|}{\|A^{-1}\|}
    \]

    Далее Храбров поломался немного, но потом починился.

    Выпишем оценку на $\|K\|$:
    \[
        \|K\| = \|Ah + \alpha(h)\|h\|\| = \|Ah - (-\alpha(h) \|h\|)\| \ge \|Ah\| - \|-\alpha(h) \|h\|\| = \|Ah\| - \|\alpha(h)\|h\|\|
    \]
    ($\|a - b\| \ge |\,\|a\| - \|b\|\,|$ - свойство нормы, но с модулем, а выше выписали сразу без модуля, ведь $|a| \ge a$)


    Продолжим оценивать:
    \[
        \|Ah\| - \|\alpha(h)\|h\|\| \ge \frac{\|h\|}{\|A^{-1}\|} - \|\alpha(h)\|\|h\| = \|h\| \left( \frac{1}{\|A^{-1}\|} - \|\alpha(h)\| \right)
    \]
    (первый переход по доказанной чуть выше оценке снизу на $\|Ah\|$)

    Храбров, когда починился, решил рассматривать только такую маленькую окрестность точки $a$, в которой $h$ -- вектора до точек из окрестности -- настолько мелкие, что $\alpha(h)$ достаточно маленькое ($\alpha(h) \to 0$ при $h \to 0$). Достаточно маленькое, чтобы $\frac{1}{\|A^{-1}\|} - \|\alpha(h)\| > 0$ было.

    Тогда получаем, что если $\|K\| \to 0$, то и $\|h\| \to 0$: так как $\frac{1}{\|A^{-1}\|} - \|\alpha(h)\| > 0$, то
    \[
        \|h\| \left( \frac{1}{\|A^{-1}\|} - \|\alpha(h)\| \right) \ge \|h\| \frac{1}{\|A^{-1}\|}
    \]
    $\|h\|$ домножается на константу, так что теперь точно должно быть ясно, что она стремится к 0, если $\|K\|$ стремится.

    В конце он вспомнил, что у нас есть непрерывность, и она всё упрощает, хотя я и не совсем осознал как.

    Продолжим: мы хотим дифференцируемость $g$ (он же $f^{-1}$), проверим (почти) определение:
    \[
        g(b+K) - g(b) = g(f(a) + f(a+h) - f(a)) - g(f(a)) = a+h -a = h = A^{-1} K - A^{-1}(\alpha(h)\|h\|)
    \]
    Пояснение: почему мы вообще куда-то $K$ подставляем и почему это нам поможет? По определению дифференцируемости, нужно, чтобы $g(b+t) = g(b) + Mt + o(\|h\|)$ при $h \to 0$. У нас $K = f(a+h) - f(a)$, притом $f$ - непрерывна, так что $a + h \to a \implies f(a+h) \to f(a) \implies K \to 0$, так что $K$ вполне подходит для проверки определения дифференцируемости: регулируя $h$ (свободную переменную), мы можем устремить $K$ к нулю.

    (Абзац выше - мои домыслы, Храбров это не проговаривал)

    Далее, откуда вылезло последнее равенство в проверке выше: \[
        K = Ah + \alpha(h)\|h\| \implies Ah = K - \alpha(h)\|h\| \implies h = A^{-1}K - A^{-1}(\alpha(h)\|h\|)
    \]

    Ого, почти то, что надо: $A^{-1}$ - дифференциал, надо лишь сделать так, чтобы $A^{-1}(\alpha(h)\|h\|) = o(\|K\|)$. Сейчас докажем:

    \[
        A^{-1}(\alpha(h)\|h\|) = o(\|K\|) \iff \frac{\|A^{-1} \alpha(h) \|h\|\|}{\|K\|} \underset{K \to 0}{\to} 0
    \]

    \[
        \frac{\|A^{-1} \alpha(h) \|h\|\|}{\|K\|} \le \frac{\|A^{-1}\| \|\alpha(h)\| \|h\| }{\|K\|} \le \frac{\|A^{-1}\| \|\alpha(h)\|\|h\|}{\|h\|\frac{1}{\|A^{-1}\|} - \|\alpha(h)\|} = \frac{\|A^{-1}\| \|\alpha(h)}{\frac{1}{\|A^{-1}\|} - \|\alpha(h)\|}
            \]

            Первый переход: свойство нормы - $\|AB\| \le \|A\|\|B\|$

            Второй переход: используем доказанную выше оценку снизу: $\|K\| \ge \|h\|\left( \frac{1}{\|A^{-1}\|} - \|\alpha(h)\|\right)$. Знаменатель уменьшим - дробь увеличится.

        Что в итоге получили: в числителе константа, умноженная на $\alpha(h) \to 0$ при $h \to 0$. А в знаменатале - что-то большее 0 (доказано выше). Так что вся дробь к 0 стремится, что нам и хотелось. Ура!

        Применим эту теорему в каждой точке - получим дифференцируемость в любой точке окрестности (области определения $f^{-1}$).

        Отмечу отдельно что $A^{-1}$ - дифференциал, при этом $A = f'(x)$, то есть $(f^{-1}(y))' = A^{-1} = (f'(x))^{-1}$, где $y = f(x)$. То есть в итоге $(f^{-1}(y))' = (f'(f^{-1}(y))^{-1}$ (подставили $x = f^{-1}(y)$)

    \end{proof}

    \begin{consequence}[Если в теореме выше $f$ - непрерывно дифференцируема, то $f^{-1}$ тоже непрерывно дифференцируема]

        \thmslashn

        Из прошлой теоремы установили, что дифференцируема, теперь хочется непрерывность производной.

        Там мы получили, что дифференциал $f^{-1}$ - это $A^{-1}$. 

        При этом по условию этого следствия, $f$ - непрерывно дифференцируемая, то есть $f'$ непрерывная, то есть $A$ - непрерывная, то есть её коэффициенты непрерывно зависят от точки (возможно, неочевидно, почему если коэффициенты непрерывны, то и матрица тоже? \ref{q25_th2.29} вот здесь норма оценивается через коэффициценты)

        Вспоминаем, как устроена обратная матрица: есть явная формула. Нам сама формула не важна (но если хочется - $A_{ij}^{-1} = \frac{(-1)^{i+j} M_{ji}}{det(A)}$, вроде так), это миноры (многочлены от коэффициентов), делёный на определитель (тоже многочлен от коэффициентов, но не нулевой, так как матрица обратима), так что это у нас композиция непрерывных функций (многочлены непрерывны, в числителе не 0), которая сама является непрерывной функцией. А это нам и надо было.
    \end{consequence}

    \begin{consequence}

        $f: D \to \R^d, D$ -- открытое, $f$ - непрерывно дифференцируема во всех точках $D$, $f'(x)$ - обратима для всех $x \in D$. Тогда для любого открытого $G \subset D$ верно, что $f(G)$ - открытое.
    \end{consequence}

    \begin{proof} \thmslashn

        Возьмём точку $b \in f(G)$. Тогда $\exists a \in D: f(a) = b$ ($a = f^{-1}(b)$). 

        Тогда применим теорему об обратной функции: сузим $f$ на $G$ - оно открытое, всё ок. И вот на этой суженной $f$ и применим теорему: 

        $\exists U \ni a$, $U \subset G$ - найдём окрестность вокруг точки $a$

        Тогда $f(U) = V$ - окрестность точки $b$, и при этом $V \subset f(G)$, так что $b \in f(U) = V \subset f(G)$, и в этой цепочке $V$ - окрестность точки $b$, то есть открытый шар. То есть для $b$ мы нашли искомый открытый шар, то есть $b$ внутренняя, то есть любая точка $f(G)$ внутренняя и потому оно открытое

        (для не открытого бы не сработало, так как мы сужали функцию на $G$, а теорема требует, чтобы $f$ действовало из открытого множества
    \end{proof}

    
    \begin{definition}

        $x \in \R^n, y \in \R^m$

        Тогда $(x,y)$ будет обозначаться вектор из $\R^{n+m} = \R^n \times \R^m$ 

        ($(x,y)$ -- состоит из 2 частей: $x \in \R^n$ и $y \in \R^m$)
    \end{definition}

    \begin{theorem} \thmslashn

        Пусть $A: \R^{n+m} \to \R^{n}$ - линейное отображение

        Тогда если $(A(h, 0_m) = 0_n \implies h = 0_n)$, то уравнение $A(x, y) = b$ имеет единственное решение

        $A(x,y) = b$

        $A(x,y) = A(x, 0) + A(0, y)$ (по линейности $A$)

        $A(x,0) = A(x,y) - A(0,y) = b - A(0, y)$

        Немного поизменяли формулу для удобства. Если решение существует и единственно для этой записи, то теорема верна.

        Покажем единственность: пусть $A(x, 0) = b - A(0, y)$ и $A(\tilde{x}, 0) = b - A(0, y)$, то $A(x - \tilde{x}, 0) = 0 \implies x - \tilde{x} = 0 \implies x = \tilde{x}$

        Покажем существование: $A(x, 0)$ - инъекция, так как ядро тривильно, т.е. $A(x, 0) = 0 \implies x = 0$. При этом $A(x, 0): \R^{n} \to \R^{n}$, то есть размерности совпадают, и потому это биекция, и потому решение есть.
    \end{theorem}

    Я так и не уверен, что такое "Образ области при невырожденном отображении" - то ли второе следствие, то ли эта теорема...
