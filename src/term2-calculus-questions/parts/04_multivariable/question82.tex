\Subsection{Билет 82: ! Частные производные высших порядков.
Теорема о перестановке частных производных в $\mathbb{R}^2$}

\begin{definition}\slashns
	
	$f: E \to \R \;\; E\subset \R^n \;\; E$ -- открытое множество.
	
	$\frac{\partial f}{\partial x_k} : E \to \R $
	
	$\frac{\partial^2 f}{\partial x_j \partial x_k} := \frac{\partial}{\partial x_j} \cdot \frac{\partial f}{\partial x_k} \;\;\; f_{x_k x_j}'' := (f'_{x_k})_{x_j}'$

	Т.е. сначала фиксируем $x_j$ (как будто параметр), считаем производную по $x_k$, затем наоборот.

	Это частная производная второго порядка, можно писать и большие аналогично.
\end{definition}

\begin{example}\slashns
	
	$f(x,y) = x^y$
	
	$\frac{\partial f}{\partial x} = yx^{y-1} \;\; \frac{\partial f}{\partial y} = x^y \ln x$
	
	$\frac{\partial^2 f }{\partial x^2} = y(y-1)x^{y-2}$
	
	$\frac{\partial^2 f}{\partial y \partial x} = \frac{\partial}{\partial y} (yx^{y-1}) = x^{y-1} + y\cdot x^{y-1} \ln x$
	
	$\frac{\partial^2 f}{\partial y^2 } = \frac{\partial}{\partial y}(x^y \ln x) = \ln^2 x \cdot x^y$
	
	$\frac{\partial^2 f}{\partial x \partial y} = yx^{y-1}\ln x + x^{y-1}$
\end{example}

\begin{example}\slashns
	
	$f(x, y) = \begin{cases}
	xy\frac{x^2 - y^2}{x^2+y^2} & x^2 + y^2 \ne 0\\
	0 & x=y=0\\
	\end{cases}$
	
	$\frac{\partial f}{\partial x} = y \frac{x^2 - y^2}{x^2+y^2} + xy \cdot \frac{-2y^2}{(x^2+y^2)^2}\cdot 2x = \frac{y(x^4-y^4) - 4x^2y^3}{(x^2 + y^2)^2}$
	
	$\frac{\partial f}{\partial x}(0, 0) = \lim\limits_{x \to 0} \frac{f(x, 0) - f(0, 0)}{x} = \lim\limits_{x \to 0} \frac{0}{x} = 0$
	
	$\frac{\partial^2 f}{\partial y \partial x}(0, 0) = \lim\limits_{y \to 0} \frac{\frac{\partial f}{\partial x}f(0, y) - \frac{\partial f}{\partial x}f(0, 0)}{y} = \lim\limits_{y \to 0} \frac{-y}{y} = -1 $
	
	Но в силу антисимметричности $x$ и $y$.
	
	$\frac{\partial^2 f}{\partial x \partial y}(0, 0) = 1$
\end{example}

\begin{theorem}\slashns
	
	$f: E \to \R \;\; E\subset \R^2 \;\; (x_0, y_0)  \in \Int E$
	
	$\frac{\partial f}{\partial x} $,
	$\frac{\partial f}{\partial y} $ и 
	$\frac{\partial^2 f}{\partial y \partial x} $ 
	существуют в окрестности точки $(x_0, y_0)$ и $\frac{\partial^2 f}{\partial y \partial x} $ ещё и непрерывна в ней
	
	
	Тогда существует и $\frac{\partial^2 f}{\partial x \partial y}$ в точке $(x_0, y_0)$. 

	Более того, $\frac{\partial^2 f}{\partial x \partial y}(x_0, y_0) = \frac{\partial^2 f}{\partial y \partial x}(x_0, y_0)$
\end{theorem}

\begin{proof}\slashns

	Рассмотрим $\phi(s) = f(s, y_0 + k) - f(s, y_0)$, что такое $k$ - поймём позже. Пока это просто какое-то число.

	Заметим, что $\phi$ дифф. в окресности точки $x_0$ (следует из существования частной производной по $y$), поэтому можем применить к ней т. Лагранжа (одномерную):
	
	$\phi(x_0 + h) - \phi(x_0) = h \phi'(x_0 + \theta_1 h) \;\; \theta_1 \in (0,1)$

	Левую часть обозначим за $\Delta$, а правую распишем через определение $\phi$, получим:

	$\Delta = h(\frac{\partial f}{\partial x} (x_0 + \theta_1 h, y_0 + k) - \frac{\partial f}{\partial x}(x_0 + \theta_1h, y_0)) $

	Обозначим $\tilde{\phi}(t) = \frac{\partial f}{\partial x} (x_0 + \theta_1 h, t)$, тогда:

	$\Delta = h(\tilde{\phi}(y_0+k) - \tilde{\phi}(y_0))$

	Снова применим лагранжа:

	$\Delta = hk\tilde{\phi}'(y_0 + \theta_2 k) = hk \frac{\partial^2 f}{\partial y \partial x} (x_0 + \theta_1 h, y_0 + \theta_2 k)$

	По условию эта вторая производная непрерывна в $(x_0, y_0)$, поэтому

	$\Delta = hk \left(\frac{\partial^2 f}{\partial y \partial x} (x_0, y_0 ) + \alpha(h, k)\right)$, где $\alpha(h, k)\to0$ при $(h,k)\to0$.
	
	Теперь перенесём вторую производную, поделим на $hk$, наложим модули:

	$\left|\frac{\Delta}{k}\frac{1}{h} - \frac{\partial^2 f}{\partial y \partial x} (x_0, y_0 )\right|=|\alpha(h, k)|<\eps$

	Заметим, что $\frac{\phi(x_0)}{k} = \frac{f(x_0, y_0 + k) - f(x_0, y_0)}{k}\to \frac{\partial f}{\partial y}(x_0, y_0)$ (предел по $k$)

	Аналогично $\frac{\phi(x_0+h)}{k}\to \frac{\partial f}{\partial y}(x_0+h, y_0)$

	Поэтому имеем

	$\left|(\frac{\partial f}{\partial y}(x_0+h, y_0)-\frac{\partial f}{\partial y}(x_0, y_0))\frac{1}{h} - \frac{\partial^2 f}{\partial y \partial x} (x_0, y_0 )\right|=|\alpha(h, k)|<\eps$

	Посмотрим на это выражение. Последнее слагаемое - число, а то что в скобках - формула из определения производной. Они отличаются друг от друга на какой-то малый $\eps$ при малых $h$. Получили как раз определение второй производной в нужном порядке. Значит, она существует, более того, равна $\frac{\partial^2 f}{\partial y \partial x} (x_0, y_0 )$, а это и хотели.
\end{proof}

\begin{definition}\slashns
	
	$f: E \to \R \;\; E\subset \R^n \;\; E$ -- открыто
	
	Функция $f$ называется $r$ раз непрерывно дифференцируемой или $r$-гладкой,
	
	если все частичные производные до $r$-ого порядка (включительно) существуют и непрерывны.
	
	Обозначение -- $C^r(E)$
\end{definition}

\begin{theorem}\slashns
	
	$f: E \to \R \;\; E \subset \R^n \;\; E$ -- открыто $f \in C^r(E)$
	
	$i_1, i_2, ...,i_r$ -- перестановка $j_1, j_2, ... , j_r$
	
	Тогда $\frac{\partial^r f}{\partial x_{i_1} \partial x_{i_2}...\partial x_{i_r}  } = \frac{\partial^r f}{\partial x_{j_1} \partial x_{j_2}...\partial x_{j_r}  }$
\end{theorem}

\begin{proof}\slashns
	
	Предыдущая теорема говорит, что любая транспозиция не меняет частной производной, а любая перестановка выражается через транспозиции. 

	На самом деле, это не совсем док-во. Мы ведь не доказали, что если в
	$\frac{\partial^r f}{\partial x_1... \partial x_i...\partial x_j...\partial x_{i_r}  }$ поменять $i$-ый и $j$-ый, то значение не изменится. Исправим чуть док-во.

	Заметим, что если $j=i+1$, то у нас одинаковое начало (всё до $i$ совпадает) + одинаковый хвост (всё после $j$ одинаковое) + остаётся ровно утверждение из прошлой теоремы. Поэтому, на самом деле, мы можем делать любые транспозиции $(i, i+1)$, но ими выражается любая перестановка.
\end{proof}

