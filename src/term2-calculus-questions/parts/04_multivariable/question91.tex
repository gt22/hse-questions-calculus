\Subsection{Билет 91: Теорема о неявной функци}

\begin{definition}\thmslashn
    
    Функции, задаваемые уравнениями -- неявные функции.
\end{definition}


\begin{theorem}[о неявной функции]\thmslashn
    
    $f \, : \, D \to \R^n \;\; D\subset \R^{n+m}$
    
    $D$ -- откртытое, $f$ -- непрерывно дифференцируема. $f(a, b) = 0, \; (a,b) \subset D$
    
    $A = f'(a, b)$, и если $A(h, 0) = 0$, то $h = 0$.
    
    Тогда $\exists W$ -- окрестность точки $b$ и единственная функция $g\, : \, W \to \R^n$, т.ч. $g(b) = a$, $g$ непрерывна дифференцируема, и $f(g(y), y ) = 0 \;\; \forall y \in W$
\end{theorem}


\begin{proof}\thmslashn
    
    $F\, : \, D \to \R^{n+m} \;\; F(x,y) = (f(x,y), y)$ -- непрерывно дифференцируема.
    
    $f(a+h, b+k) = f(a,b) + A(h,k) + r(h, k) = A(h,k) + r(h, k)$
    
    $F(a+h, b+k) = F(a,b) + (A(h, k), k) + (r(h, k), 0) = (0, b) + (A(h, k), k) + (r(h, k), 0)$
    
    $F(a+h, b+k) = (f(a+h, b+k), b+k)$
    
    $F'(a, b)(h, k) = (A(h, k), k)$
    
    Поймем, что $F'(a, b)$ инъекция. 
    
    Пусть $(A(h, k), k) = (0, 0) \Rightarrow k = 0$ и $A(h, 0) = 0 \Rightarrow h = 0$, значит $F'(a, b)$ инъекция.
    
    $F$ удволетворяет условиям теоремы об обратоной функции, тогда $\exists U$ -- окрестность точки $(a,b)$ и $V$ -- окрестность точки $(0,b)$, т.ч. $F \,:\, U \to V$ биекция. $G = F^{-1} \, : \, V \to Y$ непрерывно дифференцируема.
    
    $G(z, w) = (\phi(z,w), w)$, т.к. $F$ сохраняет последню коодинату.
    
    $(z, w) = F(G(z, w)) = (f(\phi(z, w), w), w) \Rightarrow f(\phi(z, w), w) = z$
    
    Возьмем $W$ -- окрестность точки $b$, т.ч. $\{b\} \times W \subset V$
    
    $g(w) := \phi (0, w) \;\; g \,:\, W \to \Re^n$
    
    $f(g(w), w) = f(\phi(0, w), w) = 0$
    
    $g(b) = \phi(0, b) = a$
    
    Доказли существование.
    
    Докажем единственность
    
    Пусть $f(x, y) = f(\~{x}, y)$, тогда $F(x, y) = F(\~{x}, y)$, но $F$ обратима, а значит $F$ -- биекция $\Rightarrow$ $x = \~{x}$
    
\end{proof}
