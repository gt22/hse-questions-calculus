\Subsection{Билет 25: Лемма Лебега. Число Лебега. Связь между компактностью и секвенциальной компактностью.}

\begin{lemma}(Лебега) \thmslashn

$K$ - секвенциальный компакт. $U_{\alpha}$ - открытое покрытие $K$. Тогда $\exists r > 0$, что $\forall x \in K \quad B_r(x)$ целиком содержится в некотором $U$ из $U_{\alpha}$.
\par $r$ называют \textbf{чилом Лебега}.

  \begin{proof} \thmslashn
   
    Пусть такого $r$ не существует. Значит ни одно $r$ не подходит. Рассмотрим последовательность $r_n = \frac{1}{n}$. И для каждого такого радиуса найдем точку $x_n$, такую что
    $B_{r_n}(x_n)$ не покрывается целиком никаким $U_{\alpha_i}$. Получили последовательность $\{x_n\}$ точек секвенциального компакта. Значит можно выбрать $x_{n_k}: \lim\limits_{k\rightarrow\infty}{x_{n_k}} = a \in K$. \par
  Найдется $\alpha_0 : a \in U_{\alpha_0}$. Так как $U_{\alpha_0}$ - открытое, то $\exists \eps > 0: B_{\eps}(a) \subset U_{\alpha_0}$. \par
  Так как $a$ - предел последовательность $x_{n_k}$, то $\exists N : \forall k \ge N \quad  \rho(x_{n_k}, a) < \frac{\eps}{2}$. К тому же если $k \ge \frac{2}{\eps} \implies n_k \ge \frac{2}{\eps} \implies \frac{1}{n_k} \le \frac{\eps}{2}$. ($n_{k} \ge k$, так-как $n_{k}$ задаёт подпоследовательность), а достаточно большое $k$ можем взять всегда. \par
  Теперь запишем цепочку вложений. 
  \[
    B_{\frac{1}{n_k}}(x_{n_k}) \subset B_{\frac{\eps}{2}}(x_{n_k}) \subset B_{\eps}(a) \subset U_{\alpha_0} 
  \]
  \begin{itemize}
    \item первое включение, потому что $\frac{1}{n_k} \le \frac{\eps}{2}$
    \item второе включение, потому что $\rho(a, x_{n_k}) < \frac{\eps}{2}$
    \item третье включение по выбору $\alpha_0$
  \end{itemize}

  Получили, что $B_{\frac{1}{n_k}}(x_{n_k}) \subset U_{\alpha_0}$ - противорчие с тем, как мы выбирали $x_{n_k}$. Значит нужный $r$ существует.

  \end{proof}

\end{lemma}

\begin{theorem} (связь компактности и секвенциальной компакнтоности) \thmslashn

  Компактность тоже самое, что и секвенциальная компактность.
  \begin{proof} \thmslashn

    Доказательство того, что если множество компактно, то оно секвенциально компактно было в предыдущем билете. \par
    Доказываем, что если множество $K$ - секвенциально компактно, то оно компактно. Рассмотрим какое-нибудь открытое покрытие $U_{\alpha}$. Для этого покрытия и $K$ возьмем $r$ - число Лебега.
  Теперь рассмотрим другое открытое покрытие $K$: $\bigcup\limits_{x\in K}{B_r(x)}$. Чтобы доказать, что $K$ - компакт, надо найти конечное подпокрытие в $U_{\alpha}$, для этого найдем конечное подпокрытие из $B_r(x_i)$ и каждый шарик накроем соответсвующим $U_i \implies$ получии коненчное подпокрытие. Осталось найти конечное подпокрытие шариками. 
  \par
  Возьмем $x_1 \in K$ и его шарик $B_r(x_1)$. Пока мы полностью не покроем $K$ будем брать \[x_i \in K\backslash\bigcup\limits_{j = 1}^{i-1}B_r(x_i)\]
  Пусть такой процесс не остановился за конечное число шагов. Значит получили последовательность точек ${x_n} \in K$. Так как $K$ - секвенциальный компкат, то из $x_{n}$ можно выбрать сходящуюся  подпоследовательность $x_{n_k}$. Но $x_{n_k}$ не является фундаментальной последовательностью, так как $\rho(x_i, x_j) \ge r$ ($r$ - размер шариков). \par
  Получили противоречие (так как из сходимости следует фундаментальность), значит процесс остановится за конечное число шагов, значит можно выбрать конечное покрытие из шариков, значит можно выбрать конечное покрытие из $U_{\alpha}$.
  \end{proof}
\end{theorem}



